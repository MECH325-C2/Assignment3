
% This LaTeX was auto-generated from MATLAB code.
% To make changes, update the MATLAB code and republish this document.

\documentclass{article}
\usepackage{graphicx}
\usepackage{color}

\sloppy
\definecolor{lightgray}{gray}{0.5}
\setlength{\parindent}{0pt}

\begin{document}

    
    
\subsection*{Contents}

\begin{itemize}
\setlength{\itemsep}{-1ex}
   \item Tapered roller bearing calculation
   \item Application-spesific variables
   \item Constants
   \item Values used by Timken manufacturer (see top of page 590)
   \item Bearing calculations, using tapered roller bearings
   \item Bearings-spesific variables
\end{itemize}


\subsection*{Tapered roller bearing calculation}

\begin{verbatim}
clc, clear,
\end{verbatim}


\subsection*{Application-spesific variables}

\begin{verbatim}
F_rA = 56.8 * 4.45 * 2.5 % [N]
F_rB = 99.8 * 4.45 * 2.5 % [N]
F_ae = 25 * 4.45 * 2.5 % [N]

R_D = 0.99; % reliability-factor

designLife = 2.5*4*250*16 % [hours]
rotationalSpeed = 120; % [rpm]

applicationFactor = 3; % a_f (see table 11-5), ASSUMPTION of machinery with moderate impact
\end{verbatim}

        \color{lightgray} \begin{verbatim}
F_rA =

  631.9000


F_rB =

   1.1103e+03


F_ae =

  278.1250


designLife =

       40000

\end{verbatim} \color{black}
    

\subsection*{Constants}

\begin{verbatim}
a = 10/3;   % for roller bearings in general
V = 1;      % as inner race rotates

 L_D = designLife * rotationalSpeed * 60    % [revolutions] desiered life
 L_R = 90*10^6;                              % [revolutions] rating life
 x_D = L_D / L_R    % dimensionaless multiple of rating life (for convenience)
\end{verbatim}

        \color{lightgray} \begin{verbatim}
L_D =

   288000000


x_D =

    3.2000

\end{verbatim} \color{black}
    

\subsection*{Values used by Timken manufacturer (see top of page 590)}

\begin{verbatim}
x_0 = 0;        % guaranteed life
b = 3/2;        % shape parameter
theta = 4.48;   % scale parameter
\end{verbatim}


\subsection*{Bearing calculations, using tapered roller bearings}

\begin{par}
See example 11-8
\end{par} \vspace{1em}


\subsection*{Bearings-spesific variables}

\begin{verbatim}
K_A = 1.5; % geometry factor for A, initial guess is 1.5
K_B = 1.5; % geometry factor for B

R_DA = sqrt(R_D) % estimate R_D for each bearing
R_DB = sqrt(R_D) % estimate R_D for each bearing

while true

    F_iA = 0.47*F_rA/K_A
    F_iB = 0.47*F_rB/K_B

    F_eA = 0.4*F_rA + K_A*(F_iB+F_ae)
    F_eB = F_rB

    % for bearing A
    C_10 = applicationFactor * F_eA * (x_D/(x_0+(theta-x_0)*(1-R_DA)^(1/b)))^(1/a) % eq 11-10, caltaloge entry C_10 should equal or exceed this value

    display(['For A: C_10 = ' num2str(C_10)]);
    K_Anew = input('New K = ');

    % for bearing B
    C_10 = applicationFactor * F_eB * (x_D/(x_0+(theta-x_0)*(1-R_DB)^(1/b)))^(1/a) % eq 11-10, caltaloge entry C_10 should equal or exceed this value

    display(['For B: C_10 = ' num2str(C_10)]);
    K_Bnew = input('New K = ');

    if (K_A == K_Anew && K_B == K_Bnew)
        break;
    else
        K_A=K_Anew;
        K_B=K_Bnew;
    end
end
\end{verbatim}

        \color{lightgray} \begin{verbatim}
R_DA =

    0.9950


R_DB =

    0.9950


F_iA =

  197.9953


F_iB =

  347.8862


F_eA =

   1.1918e+03


F_eB =

   1.1103e+03


C_10 =

   9.3211e+03

For A: C_10 = 9321.0703
\end{verbatim} \color{black}
    
        \color{lightgray} \begin{verbatim}Error using input
Cannot call INPUT from EVALC.

Error in bearings (line 56)
    K_Anew = input('New K = ');
\end{verbatim} \color{black}
    


\end{document}
    
